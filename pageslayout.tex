% \usepackage[a4paper]{geometry}
\section{Pages Layout}
\paragraph{}
\LaTeX{} and the document calss will normally take care of page layout issues for you.
For submission to an academic publication, this entire topic will be out of your hands,
as the publishers want to control the presentation.
\paragraph{}
However, for your own documents, there are some obvious settings that you may wish to
change: margins, page orientation and columns, to name but three.
\paragraph{}
The purpose of this tutorial is to show you how to configure
your pages.

\subsection{Page dimensions}
\subsubsection{Top margin above Chapter}
\subsubsection{Page size issues}



\subsection{Page orientation}
\subsection{Page styles}
\subsubsection{Standard page styles}

\paragraph{Commands:}
\begin{description}
	\item[\texttt{$\backslash$pagestyle\{}\emph{style}\texttt{\}}:] set globle pages style.
	\item[\texttt{$\backslash$thispagestyle\{}\emph{style}\texttt{\}}:] set this page style.
\end{description}

\paragraph{Available \emph{style}s:}
\begin{description}
	\item[\emph{empty}:] Both header and footer are clear
	\item[\emph{plain}:] Header is clear, but the footer contins the page number in the center
	\item[\emph{headings}:] Footer is blank, header displays information according to documents class (e.g., section
		name) and page number top right
	\item[\emph{myheadings}:] Page number in top right, and it is possible to control the rest of the header. Available commands:
		\begin{description}
			\item[\texttt{$\backslash$markright}:] standard document class, book, report, and article
			\item[\texttt{$\backslash$markboth}:] only in the book class
		\end{description}
\end{description}

\paragraph{Example:} look at this page's top.
\begin{verbatim}
\thispagestyle{myheadings}
\markright{xuansamdinh \hfill n2i \hfill}
\end{verbatim}

\thispagestyle{myheadings}
\markright{xuansamdinh \hfill n2i \hfill}

\paragraph{\emph{nopageno} package:} This package will make \verb|\pagestyle{plain}| have the same effect as
\verb|\pagestyle{empty}|, effectively suppressing page numbering when it is used.

\subsubsection{Customising with \texttt{fancyhdr} package}

\paragraph{}Add following line to your preamble:
\begin{verbatim}
\usepackage{fancyhdr}
\setlength{\headheihgt}{15.2pt}
\pagestyle{fancy}
\end{verbatim}

\subsubsection{Another approach with \texttt{fancyhdr}}
\subsubsection{Page \emph{n} of \emph{m}}
\paragraph{}
Some people like to pu the current page nummber in context with the whole document.
\LaTeX{} only provides to current page number. However, you can use the \emph{lastpage}
package to find the total number of pages, like this:
\begin{verbatim}
\usepackage{lastpage}
. . .
\cfoot{\thepage\ of \pageref{LastPage}}
\end{verbatim}

\paragraph{}
\emph{Note the capital letters}. Also, add a backslash(\texttt{$\backslash$}) after
\verb|\thepage| to ensure \underline{adequate} space between the page number and
`of'. And recall, when using references, that you have to run \LaTeX{} an extra time
to resolve the cross-references.

\subsection{Multi-column pages}
\paragraph{Simply ways}
\begin{verbatim}
\documentclass[twocolumn]{article}
\end{verbatim} 
\paragraph{\emph{multicol} package:}much more useful for handling multiple columns.
It has several \underline{advantages}:
\begin{itemize}
	\item[-] Can support up to ten columns.
	\item[-] Implements a \emph{mutilcols} environment, therefore, it is possible to
		mix the number of columns with in a document.
	\item[-] Additionally, the environment can be nested inside other environment, such as figure.
	\item[-] \texttt{Multicol} outputs \emph{balanced} columns, whereby the columns on the
		final page will be of roughly equal length.
	\item[-] Vertical rules between columns can be customised.
	\item[-] Column environment can be easily customised locally or globally.
\end{itemize}
Floats are not fully supported by this environment. It can only cope if you use the starred
form of the float commands (e.g., \verb|\begin{figure*}|) which makes the float span all columns.
This is not hugely problematic, since floats of the same width as a column may be too small, and
you would probably want to span them anyway.
\paragraph{}
To create a typical two-column layout:
\begin{verbatim}
\begin{multicols}{2}
	lots of text
\end{multicols}
\end{verbatim}

\paragraph{}
The parameter \verb|\columnseprule| holds the width of the vertical rules. By default, the lines
are omitted as this parameter is set to a \underline{length} of 0pt. \\
Change the horizontal space in betwen columns which \verb|\columnsep| parameter.
\begin{description}
	\item[\texttt{$\backslash$columnseprule}:] holds the width of the vertical rules, default is 0pt.
		\begin{verbatim}
		\setlength{\columnseprule}{1pt}
		\end{verbatim}
	\item[\texttt{$\backslash$columnsep}:] horizonal space between columns, default is 10pt.
		\begin{verbatim}
		\setlength{\columnsep}{20pt}
		\end{verbatim}
\end{description}

\subsection{Manual page formatting}

\paragraph{Commands:} List of all command are in the table [\ref{manuallyformat}]

\paragraph{\texttt{number}} is the priority of the command in range from \emph{0} to \emph{4}.
\begin{description}
	\item[\emph{0}:] it will be easily ignored
	\item[\emph{4}:] do it anyway
\end{description}
\begin{table}[!htbp]
	\centering
	\begin{tabular}{ l @{ } p{7.5cm} }
		\toprule
		\textbf{Command} & \textbf{Description} \\ \midrule
		\verb|\newline| & Breaks the line at the point of the command \\ \hline
		\verb|\\| & Shorter version of the \verb|\newline| command \\ \hline
		\verb|\\*| & Breaks the line at the point of the command and additionally
		prohibits a page break after the forced line break \\ \hline
		\verb|\linebreak[number]| & Breaks the line at the point of the command.
		\texttt{number} is in range from \emph{0} $\rightarrow$ \emph{4} \\ \hline
		\verb|\newpage| & Ends the current page and starts a new one \\ \hline
		\verb|\pagebreak[number]| & Breaks the current page at the point of the command \\ \hline
		\verb|\nopagebreak[number]| & Stop the page being broken at the point of the command \\ \hline
		\verb|\clearpage| & Ends the current page and causes any floats encountered in the input, but
		yet to appear, to be printed \\ \bottomrule
	\end{tabular}
	\caption{Manually page formatting}
	\label{manuallyformat}
\end{table}

\subsection{Windows and orphans}

\paragraph{}
Put these commands in document preamble:
\begin{verbatim}
\windowpenalty=300
\clubpenalty=300
\end{verbatim}
Try increasing these values if this does not help.

\paragraph{}
Have rubber band values for the space between paragraphs:
\begin{verbatim}
\setlength{\parskip}{3ex plus 2ex minus 2ex}
\end{verbatim}

\subsection{Summary}
\subsection{References}
