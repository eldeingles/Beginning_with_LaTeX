\section{Các môi trường soạn thảo}
\verb|\documentclass{article}|\\
\verb|\begin{document}|
\begin{center}
Đây là nơi nội dung soạn thảo được xử lý và hiển thị\footnote{Một ví dụ về footnote}.\\
\ldots
\end{center}
\verb|\end{document}|
%{{{
\section{Preamble}
\subsection{Tiêu đề, tác giả, ngày tháng}
\verb|\title{Beginning with \LaTeX}| --- Tiêu đề của tài liệu.\\
\verb|\author{xuansamdinh}| --- Tác giả.\\
\verb|\date{Nov 26, 2011}| --- Ngày tạo.

\subsubsection{Lưu ý:}
\begin{itemize}
\item[-] \verb|\LaTeX| là cụm từ được định nghĩa trước, và sẽ in ra ``\LaTeX''.
\item[-] Bạn có thể viết: \verb|\title{\textbf{Beginning with \LaTeX}}| để in ra
tiêu đề có dạng: ``\textbf{Beginning with \LaTeX}''
\end{itemize}

%{{{

\section{Bắt đầu tài liệu}
\begin{verbatim}
\begin{document} <-- Bắt đầu soạn thảo.
\maketitle       <-- Tạo tiêu đề.
\tableofcontents <-- Tảo bảng mục lục.
\end{document}   <-- Kết thúc soạn thảo.
\end{verbatim}
\subsection{Tóm tắt nội dung:}
\verb|\renewcommand{\abstractname}{--- Philosophy ---}|\\
\verb|\begin{abstract}|\\
... Đây là mục tóm tắt nội dung của tài liệu ...\\
\verb|\end{abstract}|
\paragraph{}
Kết quả sẽ tương tự như sau:
\begin{abstract}
Một tài liệu ngắn về bước đầu tìm hiểu \LaTeX
\end{abstract}
\paragraph{}
Sử dụng câu lệnh:\\
\verb|\renewcommand{\abstractname}{Tên cần đổi thành}|\\
để thay đổi dòng chữ mặc định là:\\
\verb|Tóm tắt nội dung| $\rightarrow$ \verb|Tên cần đổi thành|
\paragraph{}
Ngoài ra còn có:\\*
\verb|\renewcommand{\contentsname}{Nội dung}|\\
\verb|\nenewcommand{\listtablename}{Danh sách các bảng biểu}|\\
\verb|\renewcommand{\listfigurename}{Danh sách các hình}|
%{{{
\section{Phần, mục}
\subsection{Levels}
\paragraph{}
\LaTeX có 7 cấp để chia phần, mục tài liệu, bao gồm:
\\*
\verb|\part{part}|\\
\verb|\chapter{chapter}|\\
\verb|\section{section}|\\
\verb|\subsection{subsection}|\\
\verb|\subsubsection{subsubsection}|\\
\verb|\paragraph{paragraph}|\\
\verb|\subparagraph{subparagraph}|
%{{{
\section{Ví dụ: Hello world!}
\begin{itemize}
	\item[-] \verb|Hello World!| trong C:
%\newpage
\begin{verbatim}
#include <stdio.h>
#include <stdlib.h>

int
main (int argc, char *argv[]) {
    printf ("Hello world!\n");
    exit(0);
}
\end{verbatim}
\item[-] \verb|Hello world!| trong bash:
\begin{verbatim}
#!/usr/bin/env bash
echo "Hello world!"
\end{verbatim}

\item[-] \verb|Hello world!| trong python:
\begin{verbatim}
>>> print "Hello world!"
Hello world!
\end{verbatim}
\end{itemize}
