\documentclass[11pt,a4paper]{article}
\usepackage{eqparbox}
\usepackage{array}
\usepackage{multirow}
\usepackage{booktabs}
%\usepackage[table]{xcolor}
%\usepackage[utf8]{vntex}
\begin{document}
\tableofcontents
\listoftables
\pagebreak


\section{Tables}
\subsection[\texttt{tabular}]{The \texttt{tabular} enviroment}
\begin{verbatim}
\begin{tabular}{pos}{table spec}
\end{verbatim}

\subsubsection{\emph{\{pos\}}}
\begin{table}[htbp]
	\centering
\begin{center}
	\begin{tabular}{| c | p{3cm} |}
		\hline
		\texttt{t} & top \\ \hline
		\texttt{c} & center (default) \\ \hline
		\texttt{b} & bottom \\ \hline
	\end{tabular}
\end{center}
	\caption{\emph{\{pos\}} options}
	\label{tabular_positions}
\end{table}
\subsubsection{\emph{\{table spec\}}}

\begin{table}[htbp]
	\centering
	\begin{tabular}{| l | p{10cm} |}
	\hline
	\texttt{l} & left-justified column \\ \hline
	\texttt{c} & center column \\ \hline
	\texttt{r} & right-justified column \\ \hline
	\verb|p{width}| & paragraph column with text vertically aligned at the top \\ \hline
	\verb|m{width}| & paragraph column with text vertically aligned at the middle \\ \hline
	\verb|b{width}| & paragraph column with text vertically aligned at the bottom \\ \hline
	\texttt{|} & vertical line \\ \hline
	\texttt{||} & double vertical line \\
	\hline
	\end{tabular}
	\caption{\emph{\{table spec\}} options.}
	\label{tabular_specs}
\end{table}

\subsubsection{Commands}
\begin{table}[htbp]
	\centering
	\begin{tabular}{| l | p{10cm} |}
	\hline
	\verb|&| & column seperator \\ \hline
	\verb|\\| & start new row. Additional space may be specified like this: \verb|\\[10pt]| \\ \hline
	\verb|\hline| & horizontal line \\ \hline
	\verb|\newline| & start a new line within a cell \\ \hline
	\verb|\cline| & partial horizontal line beginning in column \emph{i} and ending in column \emph{j} \\ \hline
	\end{tabular}
\caption{List of \emph{command}s.}
\label{tabular_commands}
\end{table}

\subsection{Column specification}
Column specification using \verb|>{\cmd}| and \verb|<{\cmd}|.

\subsection[Text wrapping]{Text wrapping in tables}
\paragraph{}
\textbf{\LaTeX{}}'s algorithms for formatting tables have a few shortcomings.
One is that it will not automatically wrap text in cells, even if it overruns
the width of the page.\\
For columns that you know will contain a certain amount
of text, then it is recommended that you use the \emph{p} attribute and specify
the desired width of the column (although it may take some trial-and-error to
get the result you want). Use the \emph{m} attribute to have the lines aligned
toward middle of the box and the \emph{b} attribute to align along the bottom
of the box.
\subparagraph{Examples}
\subsection[Text justfication]{Text justification in tables}
\paragraph{}
The \verb|tabular| enviroment helps control where lines should break, but
cannot justify the text, which leads to ragged right edges.\\
The \texttt{eqparbox} package provides the command \verb|\eqmakebox| which is
like \verb|\makebox| but instead of a \texttt{width} argument, it takes a tag.

\subsection{Other environments inside tables}
\paragraph{}
To resolve the problems which occurse when using other environments such as
\texttt{verbatim} or \texttt{enumerate} inside table cells: change column specifier
- \emph{\{table specs\}}, to ``\emph{paragraph}'' (\texttt{p}, \texttt{m} or \texttt{b}),
i.e:
\begin{verbatim}
\begin{tabular}{| c | m{5cm}|}
\end{verbatim}

\subsection[Multiple columns]{Defining multiple columns}
\paragraph{}
It is possible to define many identical columns at once using the \\
\verb|*{num}{ptr}|\\ syntax.\\
I.e. A table contains 6 centered columns:
\begin{verbatim}
\begin{tabular}{| l | *{6}{c} | r |}
\end{verbatim}
\begin{table}[htbp]
	\centering
	\begin{tabular}{| l | *{6}{c} | r |}
	Team & P & W & D & L & F & A & Points \\ \hline
	MU & 6 & 4 & 0 & 2 & 10 & 5 & 12 \\
	\end{tabular}
\caption{Multiple column}
\label{multicols}
\end{table}
\subsection{\texttt{@}-expressions}
\paragraph{}
The column seperator can be specified with the \verb|@{..}| contruct.\\
This command replaces the symbol ``\verb:|:'' by the \texttt{@}\{\emph{symbol}\} that was defined.

\paragraph{}
To add space, use \verb|@{\hspace{|\emph{width}\verb|}}|%\verb|@{\hspace{width}}|



\subsection{Spanning}
\subsubsection{Rows spanning multiple columns}
\paragraph{}
Command: \verb|\multicolumn{|\emph{num\_cols}\verb|}{|\emph{alignment}\verb|}{|\emph{contents}\verb|}|.
\begin{description}
	\item[\emph{num\_cols}:] number of subsequent columns to merge.
	\item[\emph{alignment}:] \texttt{l}, \texttt{c}, \texttt{b} or \verb|p{|\emph{width}\verb|}| and so on.
	\item[\emph{contents}:] actual data.
\end{description}

\begin{verbatim}
	\begin{tabular}{| l | l |}
		\hline
		\multicolumn{2}{| c |}{List of Words}\\ \hline
		n2i & nothing is impossible \\
		xsd & xuansamdinh \\
		vnluser & vietnam linux user \\ \hline
	\end{tabular}
\end{verbatim}
\begin{table}[htbp]
	\centering
	\begin{tabular}{| l | l |}
		\hline
		\multicolumn{2}{| c |}{List of Words} \\ \hline
		n2i & nothing is impossible \\
		xsd & xuansamdinh \\
		vnluser & vietnam linux user \\ \hline
	\end{tabular}
	\caption{Rows spanning multi cols}
	\label{rowsspan}
\end{table}
\subsubsection{Columns spanning multiple rows}

\paragraph{}
The first thing to do: \verb|\usepackage{|\emph{multirow}\verb|}|.\\
And then using the command:\\ \verb|\multirow{|\emph{num\_rows}\verb|}{|\emph{width}\verb|}{|\emph{contents}\verb|}|.
\begin{description}
	\item[\{*\} for the \{\emph{width}\}] means the contents' natural width.
\end{description}

\paragraph{Note:} The blank entry must be inserted fo each appropriate cell in each subsequent row to be spanned.

\subsubsection{Spanning in both directions simultaneously}
\begin{verbatim}
\begin{tabular}{cc | c | c | c | c | l}
	\cline{3-6}
	& & \multicolumn{4}{| c |}{Primes} \\
	\cline{3-6}
	& & 2 & 3 & 5 & 7 \\
	\cline{1-6}
	\multicolumn{1}{| c |}{\multirow{2}{*}{Powers}} &
	\multicolumn{1}{| c |}{504} &
	3 & 2 & 0 & 1 & \\
	\cline{2-6}
	\multicolumn{1}{| c |}{} &
	\multicolumn{1}{| c |}{540} &
	2 & 3 & 1 & 0 & \\
	\cline{1-6}
	\multicolumn{1}{| c |}{\multirow{2}{*}{Powers}} &
	\multicolumn{1}{| c |}{gcd} &
	2 & 2 & 0 & 0 & min \\
	\cline{2-6}
	\multicolumn{1}{| c |}{} &
	\multicolumn{1}{| c |}{lcm} &
	3 & 3 & 1 & 1 & max \\
	\cline{1-6}
\end{tabular}
\end{verbatim}

\begin{table}[htbp]
	\centering
	\begin{tabular}{cc | c | c | c | c | l}
	\cline{3-6}
	& & \multicolumn{4}{| c |}{Primes} \\
	\cline{3-6}
	& & 2 & 3 & 5 & 7 \\
	\cline{1-6}
	\multicolumn{1}{| c |}{\multirow{2}{*}{Powers}} &
	\multicolumn{1}{| c |}{504} & 3 & 2 & 0 & 1 & \\
	\cline{2-6}
	\multicolumn{1}{| c |}{} &
	\multicolumn{1}{| c |}{540} &
	2 & 3 & 1 & 0 & \\
	\cline{1-6}
	\multicolumn{1}{| c |}{\multirow{2}{*}{Powers}} &
	\multicolumn{1}{| c |}{gcd} & 2 & 2 & 0 & 0 & min \\
	\cline{2-6}
	\multicolumn{1}{| c |}{} &
	\multicolumn{1}{| c |}{lcm} &
	3 & 3 & 1 & 1 & max \\
	\cline{1-6}
	\end{tabular}
	\caption{Columns multiple rows}
	\label{multirows}
\end{table}

\subsection{Resize tables}

\paragraph{Require:} The \emph{graphicx} package.
\paragraph{Command:} \verb|\resizebox{|\emph{width}\verb|}{|\emph{height}\verb|}{|\emph{object}\verb|}|.
\begin{verbatim}
\resizebox{10cm}{!}{\begin{tabular} . . . \end{tabular}}
\end{verbatim}
\paragraph{Alternatively:} \verb|\scalebox{|\emph{ratio}\verb|}{|\emph{object}\verb|}|. Using in the same way, but use \emph{ratio}
instead of \emph{fixed width}.
\begin{verbatim}
\scalebox{1.5cm}{ \begin{tabular} . . . \end{tabular} }
\end{verbatim}

\paragraph{Tweak space between columns.} \verb|\setlength{\tabcolsep}{5pt}|. The \LaTeX{}'s default value is \emph{6pt}.

\subsection{Sideaways tables}
\paragraph{Package:} \emph{rotating}.
\paragraph{Environment:} \emph{sideawaystable}.
\begin{verbatim}
\begin{sideawaystable}
	\begin{tabular}
		. . .
	\end{tabular}
\end{sideawaystable}
\end{verbatim}
\paragraph{Alternatively:} \emph{rotfloat} package provides the `H' options for \\
\emph{sideawaystable} environment.
\begin{verbatim}
\begin{sideawaystable}[H]
\end{verbatim}

\subsection{Alternate rows color in tables}
\paragraph{Require:} \emph{xcolor} with \emph{table} option.
\begin{verbatim}
\usepackage[table]{xcolor}
\end{verbatim}
\paragraph{Command:} \verb|\rowcolor{|\emph{starting row}\verb|}{|\emph{odd color}\verb|}{|\emph{even color}\verb|}|.
%\begin{center}
\begin{verbatim}
	\rowcolor{1}{red}{blue}
	\begin{tabular}{| c | c | c |}
		\hline
		1 & 2 & 3 \\ \hline
		4 & 5 & 6 \\ \hline
		7 & 8 & 9 \\ \hline
	\end{tabular}
\end{verbatim}
%\end{center}

\paragraph{Other commands:}
\begin{itemize}
	\item[-] \verb|\hidecolor|: deactive highlighting color of a specified row.
	\item[-] \verb|\showcolor|: reactive highlighting color of a specified row.
\end{itemize}

\subsection{Color of individual cells}
\paragraph{Require:} \emph{xcolor} with \emph{table} option.
\paragraph{Command:} \verb|\cellcolor[|\emph{gray}\verb|]{0.9}|
\begin{itemize}
	\item[-] \emph{gray}: denotes \emph{grayscale} colorscheme, not the color \emph{grey}.
	\item[-] \verb|0.9|: denotes how dark the grey is.
\end{itemize}
\subparagraph{Color the cell:} \verb|\cellcolor{|\emph{blue}\verb|}|.

\subsection{Partial vertical lines}

\subsection[\texttt{table}]{The \texttt{table} environment - captioning}
\begin{description}
	\item[Form:] \verb|\begin{table}[|\emph{placement}\verb|]|
	\item[\emph{placement}s:] default is \verb|[tbp]|\hfill
		\begin{description}
			\item[\emph{h}:] here.
			\item[\emph{t}:] top of page.
			\item[\emph{b}:] bottom of page.
			\item[\emph{p}:] float.
			\item[\emph{!}:] float.
		\end{description}
	\item[Commands:] \hfill
		\begin{description}
			\item[\texttt{$\backslash$centering}:] centering sub element.
			\item[\texttt{$\backslash$caption}:] set caption of the table.
			\item[\texttt{$\backslash$label}:] set \emph{label} to reference.
		\end{description}
\end{description}

\paragraph{Example:}
\begin{verbatim}
\begin{table}[htbp]
	\centering
	\begin{tabular}{| l | p{10cm} |}
	\hline
	hmm & Somthing is here. \\ \hline
	\end{tabular}
\caption{table example}
\label{table_examp}
\end{verbatim}

\begin{table}[htbp]
	\centering
	\begin{tabular}{| l | p{10cm} |}
	\hline
	hmm & Somthing is here. \\ \hline
	\end{tabular}
	\caption{\texttt{table} example}
	\label{table_examp}
\end{table}

\subparagraph{} Other examples are [\ref{tabular_positions}], [\ref{tabular_specs}], [\ref{tabular_commands}].

\subsection[\texttt{tabular$\star$}]{The \texttt{tabular$\star$} environment - controlling table width}
\subsubsection{Example 1}
\begin{verbatim}
	\begin{tabular*}{0.7\textwidth}{| c | c | c | r |}
		\hline
		\textbf{Midname} & \textbf{Firstname} &
		\textbf{Lastname} & \textbf{YoB} \\
		\hline
		xuan & sam & dinh & 1990 \\
		\hline
	\end{tabular*}
\end{verbatim}
\subsubsection{Example 2}


\begin{verbatim}
	\begin{tabular*}{0.75\textwidth}
		{@{\extracolsep{\fill}} | c | c | c | r |}
		\hline
		\textbf{Midname} & \textbf{Firstname} &
		\textbf{Lastname} & \textbf{YoB} \\
		\hline
		xuan & sam & dinh & 1990 \\
		\hline
	\end{tabular*}
\end{verbatim}



\begin{table}[htbp]
	\centering
	\begin{tabular*}{0.75\textwidth}{@{\extracolsep{\fill}} | c | c | c | r |}
		\hline
		\textbf{Midname} & \textbf{Firstname} & \textbf{Lastname} & \textbf{YoB} \\
		\hline
		xuan & sam & dinh & 1990 \\
		\hline
	\end{tabular*}
	\caption{Controlling table width - 2}
	\label{tableWidth2}
\end{table}

\subsection[\texttt{tabularx} package]{The \texttt{tabularx} package - simple column stretching}
\subsection{Vertically centerd images}

\paragraph{}
Inserting images into a table row will align it at the top of the cell. By using the \emph{array}
package this problem can be solved. Defining a new column type will keep the image vertically centered.
\begin{verbatim}
\newcolumntype{V}
{>{\centering\arraybackslash} m{.4\linewidth}}
\end{verbatim}

\paragraph{}
Or use a parbox to center the image:\\
\verb|\parbox[c]{1em}{\includegraphics{|\emph{image.png}\verb|}}|

\paragraph{}
A raisebox works as well, also allowing to manually fine-tune the alignment with its
first parameter: \\
\verb|\raisebox{-.5\height}{\includegraphics{|\emph{image.png}\verb|}}|

\subsection{Professional tables}

\paragraph{}
Many professionally typeset books and journals feature simple tables,
which have appropriate spacing above and below lines, and almost \emph{never}
use vertical rules. \\
Many examples of \LaTeX{} tables showcase the use of vertical rules (using ``\verb+|+''),
and double-rules (using ``\verb|\hline\hline|'' or ``\verb+||+''), which are regarded as
unnecessary and distracting in a professionally published form. The \emph{booktabs}
package is useful for easily providing this professionalism in \LaTeX{} tables, and
documentation also provides guidelines on what constitutes a ``good'' table.
\paragraph{}
In brief, the package uses \verb|\toprule| for the uppermost rule (or line),
\verb|midrule| for the rules apearing in the middle of the table (such as under the header),
and \verb|\bottomrule| for the lowermost rule. \\
This ensures that the rule weight and spacing are acceptable. In addition, \verb|\cmidrule|
can be used for mid-rules that span specified columns.\\
The following example contrasts the use of \emph{booktabs} and to quivalent normarl \LaTeX{}
implementations (the second example requires \emph{array} or \emph{dcolumn},
and the third example requires \emph{booktabs} package).
\subsubsection{Normal \LaTeX{}}
%\paragraph{Code:}
\begin{verbatim}
	\begin{tabular}{| l l r |}
		\hline
		\multicolumn{2}{c}{Item} \\
		\cline{1-2} Animal & Description & Price (\$) \\
		\hline
		Gnat & per gram & 13.65 \\
		& each & 0.01 \\
		Gnu & stuffed & 92.50 \\
		Emu & stuffed & 33.33 \\
		Armadillo & frozen & 8.99 \\
		\hline
	\end{tabular}
\end{verbatim}

%\paragraph{Result:}
\begin{table}[htbp]
	\centering
	\begin{tabular}{| l l r |}
		\hline
		\multicolumn{2}{c}{Item} \\
		\cline{1-2} Animal & Description & Price (\$) \\
		\hline
		Gnat & per gram & 13.65 \\
		& each & 0.01 \\
		Gnu & stuffed & 92.50 \\
		Emu & stuffed & 33.33 \\
		Armadillo & frozen & 8.99 \\
		\hline
	\end{tabular}
	\caption{Table using normal \LaTeX{}'s commands}
	\label{nomarllatex}
\end{table}

\paragraph{Code:}
\begin{verbatim}
	\begin{tabular}{| l l r |}
		\hline
		\multicolumn{2}{c}{Item} \\
		\cline{1-2} Animal & Description & Price (\$) \\
		\hline
		Gnat & per gram & 13.65 \\
		& each & 0.01 \\
		Gnu & stuffed & 92.50 \\
		Emu & stuffed & 33.33 \\
		Armadillo & frozen & 8.99 \\
		\hline
	\end{tabular}
\end{verbatim}

%\paragraph{Result:}
\begin{table}[htbp]
	\centering
	\begin{tabular}{| l l r |}
		\hline
		\multicolumn{2}{c}{Item} \\
		\cline{1-2} Animal & Description & Price (\$) \\
		\hline
		Gnat & per gram & 13.65 \\
		& each & 0.01 \\
		Gnu & stuffed & 92.50 \\
		Emu & stuffed & 33.33 \\
		Armadillo & frozen & 8.99 \\
		\hline
	\end{tabular}
	\caption{Table using normal \LaTeX{}'s commands}
	\label{nomarllatex}
\end{table}

\subsubsection{Using \emph{array}}
\paragraph{Code:}
\begin{verbatim}
% usepackage booktabs or dcolumn.
\begin{tabular}{  l l r  }
	\firsthline
	\multicolumn{2}{c}{Item} \\
	\cline{1-2} Animal & Description & Price (\$) \\
	\hline Gnat & per gram & 13.65 \\ &
	each & 0.01 \\
	gnu & stuffed & 92.50 \\
	Emu & stuffed & 33.33 \\
	Armadillo & frozen & 8.99 \\
	\lasthline
\end{tabular}
\end{verbatim}

%\paragraph{Result:}
\begin{table}[htbp]
	\centering
	\begin{tabular}{  l l r  }
	\firsthline
	\multicolumn{2}{c}{Item} \\
	\cline{1-2} Animal & Description & Price (\$) \\
	\hline Gnat & per gram & 13.65 \\ &
	each & 0.01 \\
	gnu & stuffed & 92.50 \\
	Emu & stuffed & 33.33 \\
	Armadillo & frozen & 8.99 \\
	\lasthline
	\end{tabular}
	\caption{Table using the \emph{array} package}
	\label{tablearray}
\end{table}



\subsubsection{Using \emph{booktabs}}
%%\documentclass[11pt,a4paper]{article}
%\begin{document}
\begin{verbatim}
\begin{tabular}{ | l l r | }
	\toprule
	\multicolumn{2}{c}{Item} \\
	\cmidrule(r){1-2} Animal & Description & Price (\$) \\
	\midrule Gnat & per gram & 13.65 \\ &
	each & 0.01 \\
	gnu & stuffed & 92.50 \\
	Emu & stuffed & 33.33 \\
	Armadillo & frozen & 8.99 \\
	\bottomrule
\end{tabular}
\end{verbatim}

\begin{tabular}{ | l l r | }
	\toprule
	\multicolumn{2}{c}{Item} \\
	\cmidrule(r){1-2} Animal & Description & Price (\$) \\
	\midrule Gnat & per gram & 13.65 \\ &
	each & 0.01 \\
	gnu & stuffed & 92.50 \\
	Emu & stuffed & 33.33 \\
	Armadillo & frozen & 8.99 \\
	\bottomrule
\end{tabular}
%\end{document}


\paragraph{Code:}
\begin{verbatim}
\begin{tabular}{  l l r  }
	\toprule
	\multicolumn{2}{c}{Item} \\
	\cmidrule(r){1-2} Animal & Description & Price (\$) \\
	\midrule Gnat & per gram & 13.65 \\ &
	each & 0.01 \\
	gnu & stuffed & 92.50 \\
	Emu & stuffed & 33.33 \\
	Armadillo & frozen & 8.99 \\
	\bottomrule
\end{tabular}
\end{verbatim}

%\paragraph{Result:}
\begin{table}[htbp]
	\centering
	\begin{tabular}{  l l r  }
	\toprule
	\multicolumn{2}{c}{Item} \\
	\cmidrule(r){1-2} Animal & Description & Price (\$) \\
	\midrule Gnat & per gram & 13.65 \\ &
	each & 0.01 \\
	gnu & stuffed & 92.50 \\
	Emu & stuffed & 33.33 \\
	Armadillo & frozen & 8.99 \\
	\bottomrule
	\end{tabular}
	\caption{Table using the \emph{booktabs} package}
	\label{tablebooktabs}
\end{table}

\paragraph{}
Usually the need arises for footnotes under a table (and not at the bottom of the page),
with a caption properly spaced above the table. These a addressed by the \emph{ctable} package. \\
It provides the option of a short caption given to be inserted in the list of tables,
instead of the actual caption (which may be quit long and inappropriate for the list of tables).
The \emph{ctable} package uses the \emph{booktabs} package.
%\hyphenation{commands}
\subsubsection[Adding rule space]{Adding rule space above or below \texttt{$\backslash$hline} and \texttt{$\backslash$cline} commands}

\paragraph{}
An alternative way to adjust the rule spacing is to add \\ \verb|\noalign{\smallskip}|
\\ before or after the \verb|\hline| and \verb|\cline{|\emph{i-j}\verb|}| commands:

\paragraph{Example:}
\begin{verbatim}
	\begin{tabular}{ l l r }
		\hline\noalign{\smallskip}
		\multicolumn{2}{c}{Item} \\
		\cline{1-2}\noalign{\smallskip}
		Animal & Description & Price (\$) \\
		\noalign{\smallskip}\hline\noalign{\smallskip}
		Gnat & per gram & 13.65 \\ &
		each & 0.01 \\
		Gnu & stuffed & 92.50 \\
		Emu & stuffed & 33.33 \\
		Armadillo & frozen 8.99 \\
		\noalign{\smallskip}\hline
	\end{tabular}
	\end{verbatim}

\begin{table}[htnp]
	\centering
	\begin{tabular}{ l l r }
		\hline\noalign{\smallskip}
		\multicolumn{2}{c}{Item} \\
		\cline{1-2}\noalign{\smallskip}
		Animal & Description & Price (\$) \\
		\noalign{\smallskip}\hline\noalign{\smallskip}
		Gnat & per gram & 13.65 \\ &
		each & 0.01 \\
		Gnu & stuffed & 92.50 \\
		Emu & stuffed & 33.33 \\
		Armadillo & frozen 8.99 \\
		\noalign{\smallskip}\hline
	\end{tabular}
	\caption{Adding rule space } % using \texttt{$\backslash$noalign\{$\backslash$smallskip\} }}
	\label{addingrulespace}
\end{table}

\paragraph{}
You may also specify the skip after a line explicitly using glue after the line terminator.
\begin{verbatim}
	\begin{tabular}{| l | l |}
		\hline
		Minaral & Color \\ [1cm]
		Ruby & red \\
		Sapphire & blue \\
		\hline
	\end{tabular}
\end{verbatim}
\begin{table}[htbp]
	\centering
	\begin{tabular}{| l | l |}
		\hline
		Minaral & Color \\ [1cm]
		Ruby & red \\
		Sapphire & blue \\
		\hline
	\end{tabular}
	\caption{Skip after a line}
	\label{skipafterline}
\end{table}

\subsection{Tables with different font size}

\paragraph{}
A table can be globally switched to a diffenrent font size by simply adding the
desired size command (here: \verb|\footnotesize|) after the \\
\verb|\begin{table}|\ldots statement:
\begin{verbatim}
begin{table}[h] \foornotesize
\caption{Performance at peak F-measure}
	\begin{tabular}{| r | r || c | c | c|}
		\ldots
	\end{tabular}
\end{table}
\end{verbatim}

The table caption font size is not affected. To control the caption font size,
see Caption Styles.

\subsection{Table with legend}

\paragraph{}
To add a legend to a table the \emph{caption} package can be used. With the
\emph{caption} package a \verb|\caption*{. . .}| statement can be added
besides the normal \verb|\caption{. . .}|.

\paragraph{Example:}

\begin{verbatim}
\begin{table}[htbp]
	\centering
	\begin{tabular}{| r | r || l | l |} 
		%\hline
		\toprule
		UID & GID & Login & Comment \\ % \hline
		\midrule
		0 & 0 & root & admin \\ \hline
		1000 & 1000 & n2i & xuansamdinh \\ % \hline
		\bottomrule
	\end{tabular}
	\caption{A normal caption}
	\caption*{A legend, even a table can be used
	\begin{tabular}{@{} l l @{}}
			\hline
			user: & xuansamdinh \\ \hline
		\end{tabular} }
\end{table}
\end{verbatim}
\begin{table}[htbp]
	\centering
	\begin{tabular}{| r | r || l | l |} 
		%\hline
		\toprule
		UID & GID & Login & Comment \\ % \hline
		\midrule
		0 & 0 & root & admin \\ \hline
		1000 & 1000 & n2i & xuansamdinh \\ % \hline
		\bottomrule
	\end{tabular}
	\caption{A normal caption}
	\caption*{A legend, even a table can be used
	\begin{tabular}{@{} l l @{}}
			\hline
			user: & xuansamdinh \\ \hline
	\end{tabular} }
	\label{captionstar}
\end{table}
The normal caption is needed for label and references.

\subsection{More features}

\paragraph{}
Have a look at one of the following packages:
\begin{description}
	\item[\emph{hhline}:] do whatever you want with horizontal lines
	\item[\emph{array}:] gives you more freedom on how to define column
	\item[\emph{colortbl}:] make your table more colorful
	\item[\emph{supertabular}:] for tables that need to sretch over serveral pages
	\item[\emph{longtable}:] similar to \emph{supertabular} \hfill
		\begin{description}
			\item[Note:] footnotes do not work properly in a normal tabular environment
				If you replace it with a longtable environment, footnotes work properly
		\end{description}
	\item[\emph{xtab}:] Yet another package for tables that need to span many pages
	\item[\emph{tabulary}:] modified \emph{tabular*} allowing width of columns set for equal heights
	\item[\emph{arydshln}:] creates dashed horizontal and vertical lines
	\item[\emph{ctable}:] allows for footnotes under table and properly spaced caption above (incorporates \emph{booktabs} package)
	\item[\emph{slashbox}:] create 2D tables
	\item[\emph{dcolumn}:] decimal point alignment of numeric cells with rounding
	\item[\emph{rccol}:] advanced decimal point alignment of numeric cells with rounding
	\item[\emph{speadtab}:] spread sheets allowing the use formulae
\end{description}
\subsection{Summary}









\end{document}
