\documentclass[11pt,a4paper]{article}
\usepackage{booktabs}

\begin{document}
\tableofcontents
\newpage
\section{Text formatting}
\subsection{Hyphenation}
\begin{verbatim}
\hyphenation{word list}
\end{verbatim}

\paragraph{Example:}
\begin{verbatim}
\hyphenation{FORTRAN Hy-phen-a-tion}
\end{verbatim}
\subsubsection{Keep words on one line}

\paragraph{}
\verb|\mbox{text}| command keeps several words on one line together, like this:
My phone number is \mbox{0988 249 750}.
\paragraph{}
\verb|\mbox{text}| is similar to \verb|\mbox|, but in addition there will be a visible
draw around the content, like this: My phone number is \fbox{0988 249 759}.


\subsection{Quote-marks}

\paragraph{}
This section will be covered later.

\subsection{Diacritics and accents}
\paragraph{}
Diacritics may be added to letters by placing special escaped metacharacters before the letter
that requires the diacritic. For a list of diacritic metacharacters, see \LaTeX{}\slash Accents.

\subsection{Space between Words and Sentences}

\paragraph{}
Disable the addtional space after periods can be disbled with:
\begin{verbatim}
\frenchspacing
\end{verbatim}
If you wanna turn \emph{frenchspacing} off, using:
\begin{verbatim}
\nonfrenchspacing
\end{verbatim}
command.

\paragraph{}
Make \LaTeX{} doesn't treat that \emph{periods} (.) as \emph{the end} of sentences, using:
\begin{verbatim}
\@.
\end{verbatim}
etc., using \verb|\@| in front of a period.
\paragraph{}
A tilde `$\sim$' character generates a space that cannot be enlarged and addtionally
prohibits a line break.

\subsubsection{Margin misalignment and interword spacing}
\paragraph{\emph{sloppypar} environment:} tells \LaTeX{} to adjust word spacing less trictly.
\paragraph{Example:}
\begin{sloppypar}
This is a paragraph with a very long word ABCDEFGHIJKLMNOPQRSTVWXYZ; then we have another bad thing
--- a long number 1234567890123456789.
\end{sloppypar}

\subsection{Ligatures}
\paragraph{}
Using:
\begin{itemize}
	\item[-] \verb|{}|
	\item[-] \verb|{kern0pt}|
\end{itemize}
between two letters of a \underline{combination} words.
\begin{verbatim}
\Large Not ``ff'', but ``f{}f''.
\end{verbatim}
\begin{table}[htbp]
	\centering
	\begin{tabular}{@{} c @{}}
		\hline\noalign{\smallskip}
		\Large Not ``ff'', but ``f{}f''. \\
		\hline\noalign{\smallskip}
	\end{tabular}
	\caption{Example of using ``Ligatures''}
	\label{ligature}
\end{table}

\subsection{Slash marks}
\begin{description}
	\item[/:] not allow following characters to be ``broken'' over multple lines.
	\item[\texttt{$\backslash$slash}:] allow following characters to be ``broken'' over multple lines.
\end{description}

\subsection{Emphasizing text}

\paragraph{}
Using \verb|\emph{|\emph{text}\verb|}| to output text like this \emph{text emphasized}.

\subsection{Fonts}

\paragraph{See also:} \emph{\LaTeX{}/Fonts}

\subsubsection{Font styles}
\paragraph{}
The are three main font families:
\begin{itemize}
	\itemsep1pt \parskip0pt \parsep0pt
	\item[-] \textrm{roman}
	\item[-] \textsf{sans serif}
	\item[-] \texttt{monospace}
\end{itemize}
List of all commands:
\begin{table}[htbp]
	\centering
	\begin{tabular}{ p{3.5cm}  | p{3.5cm} | p{4cm} }
		\toprule
		\textbf{\LaTeX{} command} & \textbf{Equivalen to} & \textbf{Output style} \\
		\midrule \\
		\verb|\textnormal{|\ldots\verb|}| & \verb|{\normalfont| \ldots\verb|}| & document font family \\ \hline \noalign{\smallskip}
		\verb|\emph{|\ldots\verb|}| & \verb|{\em| \ldots\verb|}| & \emph{emphasis} \\ \hline \noalign{\smallskip}
		\verb|\textrm{|\ldots\verb|}| & \verb|{\rmfamily| \ldots\verb|}| & \textrm{roman font family} \\ \hline \noalign{\smallskip}
		\verb|\sffamily{|\ldots\verb|}| & \verb|{\sffamily| \ldots\verb|}| & \textsf{sans serif ff} \\ \hline \noalign{\smallskip}
		\verb|\texttt{|\ldots\verb|}| & \verb|{\texttt| \ldots\verb|}| & \texttt{teletypefont family} \\ \hline \noalign{\smallskip}
		\verb|\textup{|\ldots\verb|}| & \verb|{\textup| \ldots\verb|}| & \textup{upright shape} \\ \hline \noalign{\smallskip}
		\verb|\textit{|\ldots\verb|}| & \verb|{\textit| \ldots\verb|}| & \textit{italic shape} \\ \hline \noalign{\smallskip}
		\verb|\textsl{|\ldots\verb|}| & \verb|{\textsl| \ldots\verb|}| & \textsl{slanted shape} \\ \hline \noalign{\smallskip}
		\verb|\textsc{|\ldots\verb|}| & \verb|{\textsc| \ldots\verb|}| & \textsc{Small Capitals} \\ \hline \noalign{\smallskip}
		\verb|\uppercase{|\ldots\verb|}| & \verb|{\uppercase| \ldots\verb|}| & \uppercase{all caps} \\ \hline \noalign{\smallskip}
		\verb|\textbf{|\ldots\verb|}| & \verb|{\textbf| \ldots\verb|}| & \textbf{bold} \\ \hline \noalign{\smallskip}
		\verb|\textmd{|\ldots\verb|}| & \verb|{\textmd| \ldots\verb|}| & \textmd{medium weight} \\
		\bottomrule
	\end{tabular}
	\caption{List of all \LaTeX{} command to specify font styles}
	\label{fontstyle}
\end{table}
\paragraph{\underline{Underline} text}
\subparagraph{\texttt{$\backslash$underline}:} make words will not be break properly.
\subparagraph{\emph{ulem} package:} provides these commands:
\begin{itemize}
	\item[-] \verb|\normalem|, anternatively, use \emph{ulem}'s \texttt{normalem} option
	\item[-] \verb|\uline{|\ldots\verb|}|
	\item[-] \verb|\uwave{|\ldots\verb|}|
	\item[-] \verb|\sout{|\ldots\verb|}|: stricke-out
\end{itemize}


\subsubsection{Sizing text}

\paragraph{Commands:} This table is not available right now.







\end{document}
